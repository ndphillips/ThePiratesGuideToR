\documentclass[]{book}
\usepackage{lmodern}
\usepackage{amssymb,amsmath}
\usepackage{ifxetex,ifluatex}
\usepackage{fixltx2e} % provides \textsubscript
\ifnum 0\ifxetex 1\fi\ifluatex 1\fi=0 % if pdftex
  \usepackage[T1]{fontenc}
  \usepackage[utf8]{inputenc}
\else % if luatex or xelatex
  \ifxetex
    \usepackage{mathspec}
  \else
    \usepackage{fontspec}
  \fi
  \defaultfontfeatures{Ligatures=TeX,Scale=MatchLowercase}
\fi
% use upquote if available, for straight quotes in verbatim environments
\IfFileExists{upquote.sty}{\usepackage{upquote}}{}
% use microtype if available
\IfFileExists{microtype.sty}{%
\usepackage{microtype}
\UseMicrotypeSet[protrusion]{basicmath} % disable protrusion for tt fonts
}{}
\usepackage[margin=1in]{geometry}
\usepackage{hyperref}
\hypersetup{unicode=true,
            pdftitle={YaRrr! The Pirate's Guide to R},
            pdfauthor={Nathaniel D. Phillips},
            pdfborder={0 0 0},
            breaklinks=true}
\urlstyle{same}  % don't use monospace font for urls
\usepackage{natbib}
\bibliographystyle{apalike}
\usepackage{color}
\usepackage{fancyvrb}
\newcommand{\VerbBar}{|}
\newcommand{\VERB}{\Verb[commandchars=\\\{\}]}
\DefineVerbatimEnvironment{Highlighting}{Verbatim}{commandchars=\\\{\}}
% Add ',fontsize=\small' for more characters per line
\usepackage{framed}
\definecolor{shadecolor}{RGB}{248,248,248}
\newenvironment{Shaded}{\begin{snugshade}}{\end{snugshade}}
\newcommand{\KeywordTok}[1]{\textcolor[rgb]{0.13,0.29,0.53}{\textbf{{#1}}}}
\newcommand{\DataTypeTok}[1]{\textcolor[rgb]{0.13,0.29,0.53}{{#1}}}
\newcommand{\DecValTok}[1]{\textcolor[rgb]{0.00,0.00,0.81}{{#1}}}
\newcommand{\BaseNTok}[1]{\textcolor[rgb]{0.00,0.00,0.81}{{#1}}}
\newcommand{\FloatTok}[1]{\textcolor[rgb]{0.00,0.00,0.81}{{#1}}}
\newcommand{\ConstantTok}[1]{\textcolor[rgb]{0.00,0.00,0.00}{{#1}}}
\newcommand{\CharTok}[1]{\textcolor[rgb]{0.31,0.60,0.02}{{#1}}}
\newcommand{\SpecialCharTok}[1]{\textcolor[rgb]{0.00,0.00,0.00}{{#1}}}
\newcommand{\StringTok}[1]{\textcolor[rgb]{0.31,0.60,0.02}{{#1}}}
\newcommand{\VerbatimStringTok}[1]{\textcolor[rgb]{0.31,0.60,0.02}{{#1}}}
\newcommand{\SpecialStringTok}[1]{\textcolor[rgb]{0.31,0.60,0.02}{{#1}}}
\newcommand{\ImportTok}[1]{{#1}}
\newcommand{\CommentTok}[1]{\textcolor[rgb]{0.56,0.35,0.01}{\textit{{#1}}}}
\newcommand{\DocumentationTok}[1]{\textcolor[rgb]{0.56,0.35,0.01}{\textbf{\textit{{#1}}}}}
\newcommand{\AnnotationTok}[1]{\textcolor[rgb]{0.56,0.35,0.01}{\textbf{\textit{{#1}}}}}
\newcommand{\CommentVarTok}[1]{\textcolor[rgb]{0.56,0.35,0.01}{\textbf{\textit{{#1}}}}}
\newcommand{\OtherTok}[1]{\textcolor[rgb]{0.56,0.35,0.01}{{#1}}}
\newcommand{\FunctionTok}[1]{\textcolor[rgb]{0.00,0.00,0.00}{{#1}}}
\newcommand{\VariableTok}[1]{\textcolor[rgb]{0.00,0.00,0.00}{{#1}}}
\newcommand{\ControlFlowTok}[1]{\textcolor[rgb]{0.13,0.29,0.53}{\textbf{{#1}}}}
\newcommand{\OperatorTok}[1]{\textcolor[rgb]{0.81,0.36,0.00}{\textbf{{#1}}}}
\newcommand{\BuiltInTok}[1]{{#1}}
\newcommand{\ExtensionTok}[1]{{#1}}
\newcommand{\PreprocessorTok}[1]{\textcolor[rgb]{0.56,0.35,0.01}{\textit{{#1}}}}
\newcommand{\AttributeTok}[1]{\textcolor[rgb]{0.77,0.63,0.00}{{#1}}}
\newcommand{\RegionMarkerTok}[1]{{#1}}
\newcommand{\InformationTok}[1]{\textcolor[rgb]{0.56,0.35,0.01}{\textbf{\textit{{#1}}}}}
\newcommand{\WarningTok}[1]{\textcolor[rgb]{0.56,0.35,0.01}{\textbf{\textit{{#1}}}}}
\newcommand{\AlertTok}[1]{\textcolor[rgb]{0.94,0.16,0.16}{{#1}}}
\newcommand{\ErrorTok}[1]{\textcolor[rgb]{0.64,0.00,0.00}{\textbf{{#1}}}}
\newcommand{\NormalTok}[1]{{#1}}
\usepackage{longtable,booktabs}
\usepackage{graphicx,grffile}
\makeatletter
\def\maxwidth{\ifdim\Gin@nat@width>\linewidth\linewidth\else\Gin@nat@width\fi}
\def\maxheight{\ifdim\Gin@nat@height>\textheight\textheight\else\Gin@nat@height\fi}
\makeatother
% Scale images if necessary, so that they will not overflow the page
% margins by default, and it is still possible to overwrite the defaults
% using explicit options in \includegraphics[width, height, ...]{}
\setkeys{Gin}{width=\maxwidth,height=\maxheight,keepaspectratio}
\IfFileExists{parskip.sty}{%
\usepackage{parskip}
}{% else
\setlength{\parindent}{0pt}
\setlength{\parskip}{6pt plus 2pt minus 1pt}
}
\setlength{\emergencystretch}{3em}  % prevent overfull lines
\providecommand{\tightlist}{%
  \setlength{\itemsep}{0pt}\setlength{\parskip}{0pt}}
\setcounter{secnumdepth}{5}
% Redefines (sub)paragraphs to behave more like sections
\ifx\paragraph\undefined\else
\let\oldparagraph\paragraph
\renewcommand{\paragraph}[1]{\oldparagraph{#1}\mbox{}}
\fi
\ifx\subparagraph\undefined\else
\let\oldsubparagraph\subparagraph
\renewcommand{\subparagraph}[1]{\oldsubparagraph{#1}\mbox{}}
\fi

%%% Use protect on footnotes to avoid problems with footnotes in titles
\let\rmarkdownfootnote\footnote%
\def\footnote{\protect\rmarkdownfootnote}

%%% Change title format to be more compact
\usepackage{titling}

% Create subtitle command for use in maketitle
\newcommand{\subtitle}[1]{
  \posttitle{
    \begin{center}\large#1\end{center}
    }
}

\setlength{\droptitle}{-2em}
  \title{YaRrr! The Pirate's Guide to R}
  \pretitle{\vspace{\droptitle}\centering\huge}
  \posttitle{\par}
  \author{Nathaniel D. Phillips}
  \preauthor{\centering\large\emph}
  \postauthor{\par}
  \predate{\centering\large\emph}
  \postdate{\par}
  \date{2017-02-21}

\usepackage{booktabs}
\usepackage{amsthm}
\makeatletter
\def\thm@space@setup{%
  \thm@preskip=8pt plus 2pt minus 4pt
  \thm@postskip=\thm@preskip
}
\makeatother

\begin{document}
\maketitle

{
\setcounter{tocdepth}{1}
\tableofcontents
}
\chapter{Preface}\label{intro}

\chapter{Getting Started}\label{started}

\chapter{Jump In!}\label{jumpin}

\chapter{The Basics}\label{basics}

\chapter{Scalers and vectors}\label{scalersvectors}

\chapter{Vector functions}\label{vectorfunctions}

\chapter{Indexing Vectors}\label{vectorindexing}

\chapter{Chapter solutions}\label{chapter-solutions}

\section{Chapter 4: The Basics}\label{chapter-4-the-basics}

\begin{enumerate}
\def\labelenumi{\arabic{enumi}.}
\setcounter{enumi}{1}
\tightlist
\item
  Which (if any) of the following objects names is/are invalid?
\end{enumerate}

\begin{Shaded}
\begin{Highlighting}[]
\NormalTok{thisone <-}\StringTok{ }\DecValTok{1}
\NormalTok{THISONE <-}\StringTok{ }\DecValTok{2}
\NormalTok{this.one <-}\StringTok{ }\DecValTok{3}
\NormalTok{This}\FloatTok{.1} \NormalTok{<-}\StringTok{ }\DecValTok{4}
\NormalTok{ThIS.....ON...E <-}\StringTok{ }\DecValTok{5}
\NormalTok{This!One!}\StringTok{ }\ErrorTok{<}\NormalTok{-}\StringTok{ }\DecValTok{6}           \CommentTok{# only this one!}
\NormalTok{lkjasdfkjsdf <-}\StringTok{ }\DecValTok{7}
\end{Highlighting}
\end{Shaded}

\begin{enumerate}
\def\labelenumi{\arabic{enumi}.}
\setcounter{enumi}{2}
\tightlist
\item
  2015 was a good year for pirate booty - your ship collected 100,000
  gold coins. Create an object called \texttt{gold.in.2015} and assign
  the correct value to it.
\end{enumerate}

\begin{Shaded}
\begin{Highlighting}[]
\NormalTok{gold.in}\FloatTok{.2015} \NormalTok{<-}\StringTok{ }\DecValTok{100800}
\end{Highlighting}
\end{Shaded}

\begin{enumerate}
\def\labelenumi{\arabic{enumi}.}
\setcounter{enumi}{3}
\tightlist
\item
  Oops, during the last inspection we discovered that one of your
  pirates Skippy McGee hid 800 gold coins in his underwear. Go ahead and
  add those gold coins to the object \texttt{gold.in.2015}. Next, create
  an object called \texttt{plank.list} with the name of the pirate
  thief.
\end{enumerate}

\begin{Shaded}
\begin{Highlighting}[]
\NormalTok{gold.in}\FloatTok{.2015} \NormalTok{<-}\StringTok{ }\NormalTok{gold.in}\FloatTok{.2015} \NormalTok{+}\StringTok{ }\DecValTok{800}
\NormalTok{plank.list <-}\StringTok{ "Skippy McGee"}
\end{Highlighting}
\end{Shaded}

\begin{enumerate}
\def\labelenumi{\arabic{enumi}.}
\setcounter{enumi}{4}
\tightlist
\item
  Look at the code below. What will R return after the third line? Make
  a prediction, then test the code yourself.
\end{enumerate}

\begin{Shaded}
\begin{Highlighting}[]
\NormalTok{a <-}\StringTok{ }\DecValTok{10}
\NormalTok{a +}\StringTok{ }\DecValTok{10}
\NormalTok{a       }\CommentTok{# It will return 10 because we never re-assigned a!}
\end{Highlighting}
\end{Shaded}

\section{Chapter 5: Scalers and
vectors}\label{chapter-5-scalers-and-vectors}

\begin{enumerate}
\def\labelenumi{\arabic{enumi}.}
\tightlist
\item
  Create the vector {[}1, 2, 3, 4, 5, 6, 7, 8, 9, 10{]} in three ways:
  once using \texttt{c()}, once using \texttt{a:b}, and once using
  \texttt{seq()}.
\end{enumerate}

\begin{Shaded}
\begin{Highlighting}[]
\KeywordTok{c}\NormalTok{(}\DecValTok{1}\NormalTok{, }\DecValTok{2}\NormalTok{, }\DecValTok{3}\NormalTok{, }\DecValTok{4}\NormalTok{, }\DecValTok{5}\NormalTok{, }\DecValTok{6}\NormalTok{, }\DecValTok{7}\NormalTok{, }\DecValTok{8}\NormalTok{, }\DecValTok{9}\NormalTok{, }\DecValTok{10}\NormalTok{)}
\end{Highlighting}
\end{Shaded}

\begin{verbatim}
##  [1]  1  2  3  4  5  6  7  8  9 10
\end{verbatim}

\begin{Shaded}
\begin{Highlighting}[]
\DecValTok{1}\NormalTok{:}\DecValTok{10}
\end{Highlighting}
\end{Shaded}

\begin{verbatim}
##  [1]  1  2  3  4  5  6  7  8  9 10
\end{verbatim}

\begin{Shaded}
\begin{Highlighting}[]
\KeywordTok{seq}\NormalTok{(}\DataTypeTok{from =} \DecValTok{1}\NormalTok{, }\DataTypeTok{to =} \DecValTok{10}\NormalTok{, }\DataTypeTok{by =} \DecValTok{1}\NormalTok{)}
\end{Highlighting}
\end{Shaded}

\begin{verbatim}
##  [1]  1  2  3  4  5  6  7  8  9 10
\end{verbatim}

\begin{enumerate}
\def\labelenumi{\arabic{enumi}.}
\setcounter{enumi}{1}
\tightlist
\item
  Create the vector {[}2.1, 4.1, 6.1, 8.1{]} in two ways, once using
  \texttt{c()} and once using \texttt{seq()}
\end{enumerate}

\begin{Shaded}
\begin{Highlighting}[]
\KeywordTok{c}\NormalTok{(}\FloatTok{2.1}\NormalTok{, }\FloatTok{6.1}\NormalTok{, }\FloatTok{6.1}\NormalTok{, }\FloatTok{8.1}\NormalTok{)}
\end{Highlighting}
\end{Shaded}

\begin{verbatim}
## [1] 2.1 6.1 6.1 8.1
\end{verbatim}

\begin{Shaded}
\begin{Highlighting}[]
\KeywordTok{seq}\NormalTok{(}\DataTypeTok{from =} \FloatTok{2.1}\NormalTok{, }\DataTypeTok{to =} \FloatTok{8.1}\NormalTok{, }\DataTypeTok{by =} \DecValTok{2}\NormalTok{)}
\end{Highlighting}
\end{Shaded}

\begin{verbatim}
## [1] 2.1 4.1 6.1 8.1
\end{verbatim}

\begin{enumerate}
\def\labelenumi{\arabic{enumi}.}
\setcounter{enumi}{2}
\tightlist
\item
  Create the vector {[}0, 5, 10, 15{]} in 3 ways: using \texttt{c()},
  \texttt{seq()} with a \texttt{by} argument, and \texttt{seq()} with a
  \texttt{length.out} argument.
\end{enumerate}

\begin{Shaded}
\begin{Highlighting}[]
\KeywordTok{c}\NormalTok{(}\DecValTok{0}\NormalTok{, }\DecValTok{5}\NormalTok{, }\DecValTok{10}\NormalTok{, }\DecValTok{15}\NormalTok{)}
\end{Highlighting}
\end{Shaded}

\begin{verbatim}
## [1]  0  5 10 15
\end{verbatim}

\begin{Shaded}
\begin{Highlighting}[]
\KeywordTok{seq}\NormalTok{(}\DataTypeTok{from =} \DecValTok{0}\NormalTok{, }\DataTypeTok{to =} \DecValTok{15}\NormalTok{, }\DataTypeTok{by =} \DecValTok{5}\NormalTok{)}
\end{Highlighting}
\end{Shaded}

\begin{verbatim}
## [1]  0  5 10 15
\end{verbatim}

\begin{Shaded}
\begin{Highlighting}[]
\KeywordTok{seq}\NormalTok{(}\DataTypeTok{from =} \DecValTok{0}\NormalTok{, }\DataTypeTok{to =} \DecValTok{15}\NormalTok{, }\DataTypeTok{length.out =} \DecValTok{4}\NormalTok{)}
\end{Highlighting}
\end{Shaded}

\begin{verbatim}
## [1]  0  5 10 15
\end{verbatim}

\begin{enumerate}
\def\labelenumi{\arabic{enumi}.}
\setcounter{enumi}{3}
\tightlist
\item
  Create the vector {[}101, 102, 103, 200, 205, 210, 1000, 1100, 1200{]}
  using a combination of the \texttt{c()} and \texttt{seq()} functions
\end{enumerate}

\begin{Shaded}
\begin{Highlighting}[]
\KeywordTok{c}\NormalTok{(}\KeywordTok{seq}\NormalTok{(}\DataTypeTok{from =} \DecValTok{101}\NormalTok{, }\DataTypeTok{to =} \DecValTok{103}\NormalTok{, }\DataTypeTok{by =} \DecValTok{3}\NormalTok{), }
  \KeywordTok{seq}\NormalTok{(}\DataTypeTok{from =} \DecValTok{200}\NormalTok{, }\DataTypeTok{to =} \DecValTok{210}\NormalTok{, }\DataTypeTok{by =} \DecValTok{5}\NormalTok{), }
  \KeywordTok{seq}\NormalTok{(}\DataTypeTok{from =} \DecValTok{1000}\NormalTok{, }\DataTypeTok{to =} \DecValTok{1200}\NormalTok{, }\DataTypeTok{by =} \DecValTok{100}\NormalTok{))}
\end{Highlighting}
\end{Shaded}

\begin{verbatim}
## [1]  101  200  205  210 1000 1100 1200
\end{verbatim}

\begin{enumerate}
\def\labelenumi{\arabic{enumi}.}
\setcounter{enumi}{4}
\tightlist
\item
  A new batch of 100 pirates are boarding your ship and need new swords.
  You have 10 scimitars, 40 broadswords, and 50 cutlasses that you need
  to distribute evenly to the 100 pirates as they board. Create a vector
  of length 100 where there is 1 scimitar, 4 broadswords, and 5
  cutlasses in each group of 10. That is, in the first 10 elements there
  should be exactly 1 scimitar, 4 broadswords and 5 cutlasses. The next
  10 elements should also have the same number of each sword (and so
  on).
\end{enumerate}

\begin{Shaded}
\begin{Highlighting}[]
\NormalTok{swords <-}\StringTok{ }\KeywordTok{rep}\NormalTok{(}\KeywordTok{c}\NormalTok{(}\StringTok{"scimitar"}\NormalTok{, }\KeywordTok{rep}\NormalTok{(}\StringTok{"broadswoard"}\NormalTok{, }\DecValTok{4}\NormalTok{), }\KeywordTok{rep}\NormalTok{(}\StringTok{"cutlass"}\NormalTok{, }\DecValTok{5}\NormalTok{)), }\DataTypeTok{times =} \DecValTok{100}\NormalTok{)}
\KeywordTok{head}\NormalTok{(swords)}
\end{Highlighting}
\end{Shaded}

\begin{verbatim}
## [1] "scimitar"    "broadswoard" "broadswoard" "broadswoard" "broadswoard"
## [6] "cutlass"
\end{verbatim}

\begin{enumerate}
\def\labelenumi{\arabic{enumi}.}
\setcounter{enumi}{5}
\tightlist
\item
  Create a vector that repeats the integers from 1 to 5, 10 times. That
  is {[}1, 2, 3, 4, 5, 1, 2, 3, 4, 5, \ldots{}{]}. The length of the
  vector should be 50!
\end{enumerate}

\begin{Shaded}
\begin{Highlighting}[]
\KeywordTok{rep}\NormalTok{(}\DecValTok{1}\NormalTok{:}\DecValTok{5}\NormalTok{, }\DataTypeTok{times =} \DecValTok{10}\NormalTok{)}
\end{Highlighting}
\end{Shaded}

\begin{verbatim}
##  [1] 1 2 3 4 5 1 2 3 4 5 1 2 3 4 5 1 2 3 4 5 1 2 3 4 5 1 2 3 4 5 1 2 3 4 5
## [36] 1 2 3 4 5 1 2 3 4 5 1 2 3 4 5
\end{verbatim}

\begin{enumerate}
\def\labelenumi{\arabic{enumi}.}
\setcounter{enumi}{6}
\tightlist
\item
  Now, create the same vector as before, but this time repeat 1, 10
  times, then 2, 10 times, etc., That is {[}1, 1, 1, \ldots{}, 2, 2, 2,
  \ldots{}, \ldots{} 5, 5, 5{]}. The length of the vector should also be
  50
\end{enumerate}

\begin{Shaded}
\begin{Highlighting}[]
\KeywordTok{rep}\NormalTok{(}\DecValTok{1}\NormalTok{:}\DecValTok{5}\NormalTok{, }\DataTypeTok{each =} \DecValTok{10}\NormalTok{)}
\end{Highlighting}
\end{Shaded}

\begin{verbatim}
##  [1] 1 1 1 1 1 1 1 1 1 1 2 2 2 2 2 2 2 2 2 2 3 3 3 3 3 3 3 3 3 3 4 4 4 4 4
## [36] 4 4 4 4 4 5 5 5 5 5 5 5 5 5 5
\end{verbatim}

\begin{enumerate}
\def\labelenumi{\arabic{enumi}.}
\setcounter{enumi}{7}
\tightlist
\item
  Create a vector containing 50 samples from a Normal distribution with
  a population mean of 20 and standard deviation of 2.
\end{enumerate}

\begin{Shaded}
\begin{Highlighting}[]
\KeywordTok{rnorm}\NormalTok{(}\DataTypeTok{n =} \DecValTok{50}\NormalTok{, }\DataTypeTok{mean =} \DecValTok{20}\NormalTok{, }\DataTypeTok{sd =} \DecValTok{2}\NormalTok{)}
\end{Highlighting}
\end{Shaded}

\begin{verbatim}
##  [1] 18.90087 19.74679 21.50177 21.52143 20.83906 19.27229 21.08989
##  [8] 21.92479 19.90739 20.00107 22.21581 19.74443 20.56699 16.60855
## [15] 21.39865 22.29092 21.51080 17.25699 19.85416 21.58654 21.50969
## [22] 21.90385 19.90779 22.35355 19.93505 17.01054 22.98062 20.73588
## [29] 21.23108 18.85386 17.78207 19.85904 22.22344 20.85090 21.55278
## [36] 18.22329 19.88227 20.49536 18.32553 17.79565 19.73841 20.48305
## [43] 20.76779 20.15649 20.35081 18.71416 21.57433 20.63154 19.43423
## [50] 18.94627
\end{verbatim}

\begin{enumerate}
\def\labelenumi{\arabic{enumi}.}
\setcounter{enumi}{8}
\tightlist
\item
  Create a vector containing 25 samples from a Uniform distribution with
  a lower bound of -100 and an upper bound of -50.
\end{enumerate}

\begin{Shaded}
\begin{Highlighting}[]
\KeywordTok{runif}\NormalTok{(}\DataTypeTok{n =} \DecValTok{25}\NormalTok{, }\DataTypeTok{min =} \NormalTok{-}\DecValTok{100}\NormalTok{, }\DataTypeTok{max =} \NormalTok{-}\DecValTok{50}\NormalTok{)}
\end{Highlighting}
\end{Shaded}

\begin{verbatim}
##  [1] -95.72500 -53.37903 -90.55670 -86.58900 -85.94617 -67.03399 -50.62943
##  [8] -80.13774 -94.26196 -50.09943 -90.59540 -52.10257 -91.47161 -95.21692
## [15] -88.13578 -91.07733 -60.09507 -99.25625 -74.19414 -71.28175 -61.74512
## [22] -51.00508 -94.47488 -57.50290 -98.43084
\end{verbatim}

\section{Chapter 4: Vector Functions}\label{chapter-4-vector-functions}

\begin{enumerate}
\def\labelenumi{\arabic{enumi}.}
\tightlist
\item
  Create a vector that shows the square root of the integers from 1 to
  10.
\end{enumerate}

\begin{Shaded}
\begin{Highlighting}[]
\NormalTok{(}\DecValTok{1}\NormalTok{:}\DecValTok{10}\NormalTok{) ^}\StringTok{ }\NormalTok{.}\DecValTok{5}
\end{Highlighting}
\end{Shaded}

\begin{verbatim}
##  [1] 1.000000 1.414214 1.732051 2.000000 2.236068 2.449490 2.645751
##  [8] 2.828427 3.000000 3.162278
\end{verbatim}

\begin{Shaded}
\begin{Highlighting}[]
\CommentTok{#or}

\KeywordTok{sqrt}\NormalTok{(}\DecValTok{1}\NormalTok{:}\DecValTok{10}\NormalTok{)}
\end{Highlighting}
\end{Shaded}

\begin{verbatim}
##  [1] 1.000000 1.414214 1.732051 2.000000 2.236068 2.449490 2.645751
##  [8] 2.828427 3.000000 3.162278
\end{verbatim}

\begin{enumerate}
\def\labelenumi{\arabic{enumi}.}
\setcounter{enumi}{1}
\tightlist
\item
  Renata thinks that she finds more treasure when she's had a mug of
  grogg than when she doesn't. To test this, she recorded how much
  treasure she found over 7 days without drinking any grogg (ie.,
  sober), and then did the same over 7 days while drinking grogg (ie.,
  drunk). Here are her results:
\end{enumerate}

\begin{table}

\caption{\label{tab:unnamed-chunk-15}Renata's treasure haul when she was sober and when she was drunk}
\centering
\begin{tabular}[t]{l|r|r}
\hline
day & sober & drunk\\
\hline
Monday & 2 & 0\\
\hline
Tuesday & 0 & 0\\
\hline
Wednesday & 3 & 1\\
\hline
Thursday & 1 & 0\\
\hline
Friday & 0 & 1\\
\hline
Saturday & 3 & 2\\
\hline
Sunday & 5 & 2\\
\hline
\end{tabular}
\end{table}

How much treasure did Renata find on average whe she was sober? What
about when she was drunk?

\begin{Shaded}
\begin{Highlighting}[]
\NormalTok{sober <-}\StringTok{ }\KeywordTok{c}\NormalTok{(}\DecValTok{2}\NormalTok{, }\DecValTok{0}\NormalTok{, }\DecValTok{3}\NormalTok{, }\DecValTok{1}\NormalTok{, }\DecValTok{0}\NormalTok{, }\DecValTok{3}\NormalTok{, }\DecValTok{5}\NormalTok{)}
\NormalTok{drunk <-}\StringTok{ }\KeywordTok{c}\NormalTok{(}\DecValTok{0}\NormalTok{, }\DecValTok{0}\NormalTok{, }\DecValTok{1}\NormalTok{, }\DecValTok{0}\NormalTok{, }\DecValTok{1}\NormalTok{, }\DecValTok{2}\NormalTok{, }\DecValTok{2}\NormalTok{)}

\KeywordTok{mean}\NormalTok{(sober)}
\end{Highlighting}
\end{Shaded}

\begin{verbatim}
## [1] 2
\end{verbatim}

\begin{Shaded}
\begin{Highlighting}[]
\KeywordTok{mean}\NormalTok{(drunk)}
\end{Highlighting}
\end{Shaded}

\begin{verbatim}
## [1] 0.8571429
\end{verbatim}

\begin{enumerate}
\def\labelenumi{\arabic{enumi}.}
\setcounter{enumi}{2}
\tightlist
\item
  Using Renata's data again, create a new vector called
  \texttt{difference} that shows how much more treasure Renata found
  when she was drunk and when she was not. What was the mean, median,
  and standard deviation of the difference?
\end{enumerate}

\begin{Shaded}
\begin{Highlighting}[]
\NormalTok{difference <-}\StringTok{ }\NormalTok{sober -}\StringTok{ }\NormalTok{drunk}

\KeywordTok{mean}\NormalTok{(difference)}
\end{Highlighting}
\end{Shaded}

\begin{verbatim}
## [1] 1.142857
\end{verbatim}

\begin{Shaded}
\begin{Highlighting}[]
\KeywordTok{median}\NormalTok{(difference)}
\end{Highlighting}
\end{Shaded}

\begin{verbatim}
## [1] 1
\end{verbatim}

\begin{Shaded}
\begin{Highlighting}[]
\KeywordTok{sd}\NormalTok{(difference)}
\end{Highlighting}
\end{Shaded}

\begin{verbatim}
## [1] 1.345185
\end{verbatim}

\begin{enumerate}
\def\labelenumi{\arabic{enumi}.}
\setcounter{enumi}{3}
\tightlist
\item
  There's an old parable that goes something like this. A man does some
  work for a king and needs to be paid. Because the man loves rice (who
  doesn't?!), the man offers the king two different ways that he can be
  paid. \emph{You can either pay me 100 kilograms of rice, or, you can
  pay me as follows: get a chessboard and put one grain of rice in the
  top left square. Then put 2 grains of rice on the next square,
  followed by 4 grains on the next, 8 grains on the next\ldots{}and so
  on, where the amount of rice doubles on each square, until you get to
  the last square. When you are finished, give me all the grains of rice
  that would (in theory), fit on the chessboard.} The king, sensing that
  the man was an idiot for making such a stupid offer, immediately
  accepts the second option. He summons a chessboard, and begins
  counting out grains of rice one by one\ldots{} Assuming that there are
  64 squares on a chessboard, calculate how many grains of rice the main
  will receive. If one grain of rice weights 1/64000 kilograms, how many
  kilograms of rice did he get? \emph{Hint: If you have trouble coming
  up with the answer, imagine how many grains are on the first, second,
  third and fourth squares, then try to create the vector that shows the
  number of grains on each square. Once you come up with that vector,
  you can easily calculate the final answer with the \texttt{sum()}
  function.}
\end{enumerate}

\begin{Shaded}
\begin{Highlighting}[]
\CommentTok{# First, let's create a vector of the amount of rice on each square:}
\NormalTok{rice <-}\StringTok{ }\DecValTok{2} \NormalTok{^}\StringTok{ }\NormalTok{(}\DecValTok{1}\NormalTok{:}\DecValTok{64}\NormalTok{)}

\CommentTok{# Here are the first few spaces}
\KeywordTok{head}\NormalTok{(rice)}
\end{Highlighting}
\end{Shaded}

\begin{verbatim}
## [1]  2  4  8 16 32 64
\end{verbatim}

\begin{Shaded}
\begin{Highlighting}[]
\CommentTok{# The result is just the sum!}
\NormalTok{rice.total <-}\StringTok{ }\KeywordTok{sum}\NormalTok{(rice)}
\NormalTok{rice.total}
\end{Highlighting}
\end{Shaded}

\begin{verbatim}
## [1] 3.689349e+19
\end{verbatim}

\begin{Shaded}
\begin{Highlighting}[]
\CommentTok{# How much does that weigh? Each grain weights 1/6400 kilograms:}
\NormalTok{rice.kg <-}\StringTok{ }\KeywordTok{sum}\NormalTok{(rice) *}\StringTok{ }\DecValTok{1}\NormalTok{/}\DecValTok{6400}
\NormalTok{rice.kg}
\end{Highlighting}
\end{Shaded}

\begin{verbatim}
## [1] 5.764608e+15
\end{verbatim}

\begin{Shaded}
\begin{Highlighting}[]
\CommentTok{# That's 5,800,000,000,000,000 kilograms of rice. Let's keep going....}
\CommentTok{# A kg of rice is 1,300 calories}

\NormalTok{rice.cal <-}\StringTok{ }\NormalTok{rice.kg *}\StringTok{ }\DecValTok{1300}
\NormalTok{rice.cal}
\end{Highlighting}
\end{Shaded}

\begin{verbatim}
## [1] 7.49399e+18
\end{verbatim}

\begin{Shaded}
\begin{Highlighting}[]
\CommentTok{# How many people can that feed for a year?}
\CommentTok{# A person needs about 2,250 calories a day, or 2,250 * 365 per year}

\NormalTok{rice.people.year <-}\StringTok{ }\NormalTok{rice.cal /}\StringTok{ }\NormalTok{(}\DecValTok{2250} \NormalTok{*}\StringTok{ }\DecValTok{365}\NormalTok{)}
\NormalTok{rice.people.year}
\end{Highlighting}
\end{Shaded}

\begin{verbatim}
## [1] 9.125102e+12
\end{verbatim}

\begin{Shaded}
\begin{Highlighting}[]
\CommentTok{# So, that amount of rice could feed 9,100,000,000,000 for a year}
\CommentTok{# Assuming that the averge lifespan is 70 years, how many lifespans could this feed?}

\NormalTok{rice.people.life <-}\StringTok{ }\NormalTok{rice.people.year /}\StringTok{ }\DecValTok{70}
\NormalTok{rice.people.life}
\end{Highlighting}
\end{Shaded}

\begin{verbatim}
## [1] 130358595868
\end{verbatim}

\begin{Shaded}
\begin{Highlighting}[]
\CommentTok{# Ok...so it could feed 130,000,000,000 (130 billion) people over their lives}

\CommentTok{# Conclusion: King done screwed up.}
\end{Highlighting}
\end{Shaded}

\begin{Shaded}
\begin{Highlighting}[]
\CommentTok{# I want to remove the white separations from the following chunk output}

\DecValTok{1+1}
\NormalTok{## [1] 2}

\DecValTok{2+4}
\NormalTok{## [1] 6}
\end{Highlighting}
\end{Shaded}

\chapter{Placeholder}\label{placeholder}

\bibliography{packages.bib,book.bib}


\end{document}
